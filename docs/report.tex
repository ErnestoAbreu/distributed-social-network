\documentclass[11pt]{article}

\usepackage[spanish]{babel}
\usepackage{graphicx}
\usepackage{hyperref}
\usepackage{amsmath, amssymb}
\usepackage{geometry}
\usepackage{setspace}
\onehalfspacing

\title{Informe - Red Social}
\author{Eduardo Brito Labrada (C411) \\ Ernesto Abreu Peraza (C412)}
\date{}

\begin{document}

\maketitle

\section{Arquitectura}

El proyecto consiste en implementar una red social distribuida para la comunicación siguiendo
la idea en la cual se base Twitter. El sistema debe brindar una infraestructura para el manejo
de usuarios, de manera que puedan publicar sus mensajes, volver a publicar mensajes que consideren
interesantes de otros usuarios y escoger quienes van a ser sus amigos. Además, como parte del 
manejo de usuarios, se debe proporcionar un mecanismo para la autenticación distribuida de manera
que se logre un modelo que no sea centralizado.

Para la arquitectura, nos basaremos en una red de servidores y clientes que se comunicana través
de gRPC. Los servidores están organizados en un anillo de Chord para la distribución y almacenamiento
de datos. A continuación se especifica en más detalle el funcionamiento de cada parte del sistema:

\begin{itemize}
    \item Cliente: se encarga de localizar un servidor activo en la red para enviar solicitudes. Al
    servidor localizado le solicita la publicación de posts, seguimiento de usuarios o autenticación.
    \item Servidor: se encarga de recibir y procesar solicitudes que le llegan de los clientes. Además,
    es quien ofrece los servicios requeridos por el sistema como publicación de posts, autenticación de 
    los usuarios y manejo de relaciones entre los usuarios.
    \item Repositorio de Datos: es el intermediario que se encarga de gestionar el almacenamiento y la
    recuperación de los datos. Se comunica con los nodos del anillo de Chord para relizar el almacenamiento
    distribuido.
    \item Nodos en el anillo de Chord: cada nodo almacena datos según el hash calculado para la llave del 
    post o usuario. Implementa las funcionalidades específicas de Chord para distribuir y localizar datos.
\end{itemize}


\end{document}
