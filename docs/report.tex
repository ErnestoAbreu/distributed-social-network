\documentclass[11pt]{article}

\usepackage[spanish]{babel}
\usepackage{graphicx}
\usepackage{hyperref}
\usepackage{amsmath, amssymb}
\usepackage{geometry}
\usepackage{setspace}
\onehalfspacing

\title{Informe - Red Social}
\author{Eduardo Brito Labrada (C411) \\ Ernesto Abreu Peraza (C412)}
\date{}

\begin{document}

\maketitle

\section{Arquitectura}

El proyecto consiste en implementar una red social distribuida para la comunicación siguiendo
la idea en la cual se base Twitter. El sistema debe brindar una infraestructura para el manejo
de usuarios, de manera que puedan publicar sus mensajes, volver a publicar mensajes que consideren
interesantes de otros usuarios y escoger quienes van a ser sus amigos. Además, como parte del 
manejo de usuarios, se debe proporcionar un mecanismo para la autenticación distribuida de manera
que se logre un modelo que no sea centralizado.

Para la arquitectura, nos basaremos en una red de servidores y clientes que se comunicana través
de gRPC. Los servidores están organizados en un anillo de Chord para la distribución y almacenamiento
de datos. A continuación se especifica en más detalle el funcionamiento de cada parte del sistema:

\begin{itemize}
    \item Cliente: se encarga de localizar un servidor activo en la red para enviar solicitudes. Al
    servidor localizado le solicita la publicación de posts, seguimiento de usuarios o autenticación.
    \item Servidor: se encarga de recibir y procesar solicitudes que le llegan de los clientes. Además,
    es quien ofrece los servicios requeridos por el sistema como publicación de posts, autenticación de 
    los usuarios y manejo de relaciones entre los usuarios.
    \item Repositorio de Datos: es el intermediario que se encarga de gestionar el almacenamiento y la
    recuperación de los datos. Se comunica con los nodos del anillo de Chord para relizar el almacenamiento
    distribuido.
    \item Nodos en el anillo de Chord: cada nodo almacena datos según el hash calculado para la llave del 
    post o usuario. Implementa las funcionalidades específicas de Chord para distribuir y localizar datos.
\end{itemize}

Los servicios estarán distribuidos en dos redes de Docker:
\begin{itemize}
    \item Red de clientes: alojará múltiples instancias de clientes que interactúan directamente con los
    usuarios, estas instancias no comparten información entre sí y dependen completamente de la red de
    servidores para procesar cada uno de los servicios del sistema.
    \item Red de servidores: contiene los contenedores que ejecutan los servicios del servidor, incluyendo
    el repositorio de datos y los nodos en el anillo de Chord. Los servidores se comunican entre sí para
    mantener la consistencia y la disponibilidad de los datos en todo momento. 
\end{itemize}

\section{Procesos}

Los principales procesos necesarios para este sistema son:
\begin{enumerate}
    \item Los clientes, que se encargan de manejar las solicitudes de los usuarios.
    \item Los servidores, que se encargan de ejecutar servicios específicos que requiere el sistema.
    \item Un repositorio de datos que actúa como intermediario para gestionar almacenamiento y recuperación
    de datos.
    \item Los nodos del anillo de Chord que se encargan de distribuir y almacenar los datos hasheados.
\end{enumerate}

Cada uno de estos servicios opera de manera asíncrona. Se utilizarán hilos para manejar las solicitudes
concurrentes, lo que permite escalabilidad y tolerancia a fallos, empleando redes de Docker separadas para
aislar la comunicación entre clientes y servidores. Esto permite un desempeño eficiente en entornos distribuidos
al garantizar una interacción fluida y segura entre las distintas componentes del sistema.

\end{document}
